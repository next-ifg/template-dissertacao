%%%%%%%%%%%%%%%%%%%%%%%%%%%%%%%%%%%%%%%%%%%%%%%%%%%%%% 
% Glossário
%%%%%%%%%%%%%%%%%%%%%%%%%%%%%%%%%%%%%%%%%%%%%%%%%%%%%% 

\begin{glossario}
  \label{glossario}

\item[Condutivímetro] - é um medidor digital portátil que mensura a condutividade elétrica do solo diretamente "\emph{in loco}".

\item[Data Logger] - é um coletor de dados também chamado de datalogger ou gravador de dados. É um dispositivo eletrônico que registra os dados ao longo do tempo ou em relação a uma localização, construído com sensores externos. São baseados em um processador digital com memórias internas para armazenamento de dados. São de uso geral para uma gama de aplicações em dispositivos de medição específicos, podem ser programáveis.

\item[GPS] - é um sistema de navegação por satélite que fornece a um aparelho receptor móvel a posição do mesmo, assim como informação horária, sob todas quaisquer condições atmosféricas, a qualquer momento e em qualquer lugar na Terra, desde que o receptor se encontre no campo de visão de quatro satélites GPS.

\item[Neossolo Regolítico] - são tipos de solos que apresentam textura arenosa e baixa capacidade de adsorção de nutrientes, quando comparado com solos argilosos, possui baixo teor de matéria orgânica e nitrogênio que diminuem, após alguns anos de uso agrícola.

\item[Nitossolo Vermelho] - são solos minerais, não-hidromórficos, apresentando cor vermelho-escura tendendo à arroxeada. São derivados do intemperismo de rochas básicas e ultrabásicas, ricas em minerais ferromagnesianos. Uma característica peculiar é que esses solos, como os Latossolos Roxos, apresentam materiais que são atraídos pelo imã. Seus teores de ferro ($Fe2O3$) são elevados (superiores a 15\%).

\item[Plintossolo Pétrico Concrecionário] - são solos que ocorrem em áreas baixas e nas bordas das chapadas, constituindo geralmente por solos pobres em nutrientes. A origem de concreções ferruginosas nos solos tem sido atribuída, de forma generalizada, às condições de variações sazonais do lençol freático. Este, inicialmente elevado, propicia a redução do ferro com a sua retirada parcial do sistema, mobilização, transporte e concentração. Posteriormente, em épocas secas, a oxidação forma plintitas constituídas por mistura de argila pobre em \emph{C} orgânico e rica em ferro e alumínio, segregada sob a forma de manchas vermelhas, que com a retirada do lençol freático, apresentam endurecimento constituindo concreções ferruginosas ou petroplintitas.

\item[PVC] - é feito a partir de repetidos processos de polimerização que convertem hidrocarbonetos, contidos em materiais como o petróleo, em um único composto chamado polímero. O vinil é formado basicamente por etileno e cloro. Por uma reação química, o etileno e o cloro combinam-se formando o dicloreto de etileno, que por sua vez é transformado em um gás chamado \emph{VCM} (Vinyl chloride monomer, em português cloreto de vinila). O passo final é a polimerização, que converte o monómero num polímero de vinil, que é o \emph{PVC}, ou simplesmente vinil, contém, em peso, $57\%$ de cloro (derivado do cloreto de sódio - sal de cozinha) e $43\%$ de eteno (derivado do petróleo).

\item[TC scan] - é uma tomografia computadorizada (\emph{TC}), originalmente apelidada tomografia axial computadorizada (\emph{TAC}), é um exame complementar de diagnóstico por imagens tridimensionais, que consiste numa representação de uma secção ou fatia do estudo. É obtida através do processamento por computador de informação recolhida após expor o objeto estudado a uma sucessão de raios $X$. Seu método principal é estudar a atenuação de um feixe de raios $X$ durante seu trajeto através de um segmento do objeto estudado; no entanto, ela se distingue da radiografia convencional em diversos elementos.

\end{glossario}

%%% Local Variables:
%%% mode: latex
%%% TeX-master: "../main-dissertacao"
%%% coding: utf-8
%%% End:
